\documentclass[a4paper,12pt]{article}
\usepackage{HomeWorkTemplate}
\usepackage{circuitikz}
\usepackage[shortlabels]{enumitem}
\usepackage{hyperref}
\usepackage{tikz}
\usepackage{amsmath}
\usepackage{amssymb}
\usepackage{tcolorbox}
\usepackage{xepersian}
\settextfont{XB Niloofar}
\usepackage{changepage}
\newcounter{subproblemcounter}
\setcounter{subproblemcounter}{1}
\newcommand{\problem}[1]
{
	\subsection*{
		تمرین
		#1
	}
}
\newcommand{\subproblem}{
	\textbf{\harfi{subproblemcounter})}\stepcounter{subproblemcounter}
}


\begin{document}
\handout
{پردازش زبان‌های طبیعی}
{احسان‌الدین عسگری}
{نیم‌سال اول 1400\lr{-}1401}
{اطلاعیه}
{امین کشیری، فاطمه توحیدیان، سید علیرضا موسوی}
{۹۷۱۰۱۰۲۶}
{تمرین سری اول}

:توضیح مربوط به انواع تایپ‌ها
\begin{enumerate}
    \item \textbf{نماد}
    
    نمادهای معاملاتی در بورس ایران
    \item \textbf{شرکت}
    
    شامل شرکت‌های بورسی ایران 
    \item \textbf{اعلان}
    
    اصطلاحات اداری همانند گزارش، اطلاعیه و ...
	\item \textbf{تحلیل}
	
	اصطلاحات خاص بورسی و تحلیلی مانند سهم رانتی، تحلیل تکنیکال، واگرایی و ...
	\item \textbf{شخصیت}
	
	نام شخصیت‌های حاضر در بازار مانند نوسان‌گیر، بازیگر و ...
	\item  \textbf{واقعه}
	
	سایر وقایع مهم بورسی که در دسته های بالا جای نگیرند مانند صف خرید، تقسیم سود، مجمع عمومی و ...
\end{enumerate}
\end{document}